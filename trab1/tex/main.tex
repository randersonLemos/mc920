\documentclass{article}

\usepackage[utf8]{inputenc}

\usepackage{geometry}
\geometry{
	a4paper,
	total={170mm,257mm},
	left=20mm,
	top=20mm,
}

\usepackage{listings}
\usepackage{xcolor}

\definecolor{codegreen}{rgb}{0,0.6,0}
\definecolor{codegray}{rgb}{0.5,0.5,0.5}
\definecolor{codepurple}{rgb}{0.58,0,0.82}
\definecolor{backcolour}{rgb}{0.95,0.95,0.92}

\lstdefinestyle{mystyle}{
	backgroundcolor=\color{backcolour},   
	commentstyle=\color{codegreen},
	keywordstyle=\color{magenta},
	numberstyle=\tiny\color{codegray},
	stringstyle=\color{codepurple},
	basicstyle=\ttfamily\footnotesize,
	breakatwhitespace=false,         
	breaklines=true,                 
	captionpos=b,                    
	keepspaces=true,                 
	numbers=left,                    
	numbersep=5pt,                  
	showspaces=false,                
	showstringspaces=false,
	showtabs=false,                  
	tabsize=2
}

\lstset{style=mystyle}

%\usepackage{indentfirst}
\setlength{\parindent}{1.5cm}% too much in my eyes delete this
% line and use the default ...


\title{
	Escondendo uma mensagem em uma imagem (trabalho 1) \\
	\Large Introdução ao Processamento de Imagem Digital \\
	Randerson A. Lemos (103897)
	2022-1S
}

\date{\vspace{-5ex}}

\begin{document}
  \pagenumbering{gobble}
  \maketitle

\section{Introdução}
A esteganografia é a área de conhecimento que se dedica a estudar técnicas de ocultação de informações (por exemplo, mensagens) em imagens sem que estas sofram alterações perceptíveis ao olho humano. A esteganografia, neste trabalho, será aplicada por meio da substituição dos bits menos significativos dos pixels das imagens escolhidas pelos bits da mensagem de interesse que se deseja ocultar. A seguir teremos três seções, a de Solução, a de Resultado e a de Conclusão. Na seção de Solução, detalhes técnicos, de usabilidade e de decisões da solução proposta são fornecidos. Na seção de Resultado, os principais resultados são apresentados. Na seção de Conclusão, há não apenas a apresentação de conclusões, mas também de discussões.

\section{Solução}
A solução utiliza a linguagem de programação Python e conta com o auxílio do gerenciador de projetos e pacotes Conda. Assumindo que o usuário tenha o Conda instalado em sua máquina, a configuração do projeto pode ser feita pela execução do comando \lstinline{conda env create -f environment.yml} a partir da pasta \textbf{trab1}. Esse comando cria o ambiente de trabalho \textbf{mc920-trab1} e instala os seguintes módulos: opencv, numpy, scipy, pandas, matplotlib. Finalizada a configuração do ambiente de trabalho em questão, o usuário deve executar o comando \lstinline{source source.sh}\footnote{O comando que configura o ambiente de trabalho mc920-trab1 precisa ser executado apenas um vez. Assim sendo, depois que este ambiente está configurado, o usuário precisa apenas executar o comando \lstinline{source source.sh}} para carregar as variáveis de ambiente adequadas e, assim, poder usar os programas do projeto dentro do próprio ambiente de trabalho recém configurado. 

Dos arquivos presentes na pasta do projeto \textbf{mc920-trab1}, destacam-se as pastas \textbf{png}, \textbf{tex}, \textbf{txt} e os programas \textbf{codificar.py}, \textbf{decodificar.py}, \textbf{mostrar\_planos.py}. A pasta \textbf{png} contém imagens no formato png que podem ser utilizadas como receptáculos da mensagem a ser ocultada. A pasta \textbf{txt} contém exemplos de textos que podem ser utilizados no processo de ocultação de suas mensagens. A pasta \textbf{tex} contém os arquivos Latex deste relatório. As informações pertinentes dos programas \textbf{codificar.py}, \textbf{decodificar.py}, \textbf{mostrar\_planos.py} serão detalhadas a seguir.

\subsection{Codificar.py}
O programa \textbf{codificar.py} é responsável por ocultar uma mensagem de interesse em uma imagem escolhida que deve ser colorida e estar no formato png. Para ser executado, esse programa precisa receber os parâmetros \textbf{imagem\_entrada}, \textbf{texto\_entrada}, e \textbf{planos\_bits}:


\begin{itemize}
	\item ao parâmetro \textbf{imagem\_entrada} deve-se fornecer o nome do arquivo da imagem a ser utilizada como `bau' da mensagem que se deseja esconder;
	\item ao parâmetro \textbf{texto\_entrada} deve-se fornecer o nome do arquivo do texto que contém a mensagem que
	se deseja ocultar;
	\item ao parâmetro \textbf{planos\_bits} deve-se fornecer os planos de bits menos significativos dos pixels da imagem escolhida que serão utilizados para registrar a mensagem de interesse. Os valores esperados para esse parâmetro são: 0, ou 1, ou 2, ou combinações desses valores separados por ‘:’. Quando mais de um plano de bits são passados ao programa pelo parâmetro \textbf{planos\_bits}, ocorre a ordenação em ordem crescente desses planos de modo que a utilização dos bits dos pixels da imagem se dê sempre do plano de bits menos significativo para o mais significativo.
\end{itemize}

\noindent
O texto de entrada que contem a mensagem a ser escondida na imagem escolhida deve apresentar um código de \textit{end of file} para que o processo subsequente de decodificação consiga identificar o final de mensagem. Aqui, o código escolhido é \textbf{!@\#FIM\#@!}. A combinação dos caracteres desse código foi pensada de modo que uma possível aparição não intencional de tal código ao longo do texto seja significativamente improvável.

Exemplos de como utilizar o programa \textbf{codificar.py} utilizando os recursos contidos dentro do próprio projeto são:

\lstinline{python3 codificar.py -imagem_entrada=png/watch.png -texto_entrada=txt/texto1.txt -planos_bits=2};

\lstinline{python3 codificar.py -imagem_entrada=png/watch.png -texto_entrada=txt/texto1.txt -planos_bits=1:2};

\lstinline{python3 codificar.py -imagem_entrada=png/watch.png -texto_entrada=txt/texto1.txt -planos_bits=0:1:2} 

\noindent
Após executado, o programa \textbf{codificar.py} gerá uma imagem de saída. Esse imagem nada mais é que a imagem de entrada contendo a mensagem do texto de entrada oculta nos seus bits menos significativos de acordo com os planos de bits passados pelo usuário. Essa imagem de saída é salva automaticamente na pasta \textbf{out} que está dentro da pasta do projeto \textbf{trab1}. Se a imagem de entrada tem o nome \textbf{img\_ent} a imagem de saída apresentará o nome \textbf{img\_ent\underline{m\_planoX}}, onde o letra X é um \textit{placeholder} para os números dos planos de bits escolhidos pelo usuário.

\subsection{Mostrar\_planos.py}
\lstinline{python3 mostrar_planos.py -imagem_entrada=out/watchm_plano12.png -planos_bits=0:1:2};

\end{document}